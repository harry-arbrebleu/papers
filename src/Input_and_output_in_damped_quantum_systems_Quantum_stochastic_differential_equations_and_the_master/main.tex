\begin{boxnote}
      \begin{description}
        \item[種類] Article
        \item[閲覧日] 13th May 2025
        \item[キーワード] 
        \item[文献番号] \cite{PhysRevA.31.3761}
        \item[関連論文] 特になし.
      \end{description}
    \end{boxnote}
    \section{量子Langevin方程式}
      着目系$\hat{H}_{\r{sys}}$,熱浴$\hat{H}_{\r{B}}$,それらの相互作用$\hat{H}_{\r{int}}$の全体を考える.
      全体のハミルトニアンは,
      \begin{align}
        \hat{H} = \hat{H}_{\r{sys}} + \hat{H}_{\r{B}} + \hat{H}_{\r{int}}
      \end{align}
      である.
      熱浴はBoson場であるとして,熱浴のハミルトニアンと相互作用のハミルトニアンは,
      \begin{align}
        \hat{H}_{\r{B}}(t) &\coloneqq \int_{-\infty}^{\infty}\dd{\omega} \hbar\omega \hat{b}^{\dag}(\omega, t)\hat{b}(\omega, t) \\ 
        \hat{H}_{\r{int}}(t) &\coloneqq \int_{-\infty}^{\infty}\dd{\omega} \hbar\i\kappa(\omega)\qty(\hat{b}^{\dag}(\omega, t)\hat{c}(t) - \hat{c}^{\dag}(t)\hat{b}(\omega, t))
      \end{align}
      とする.
      $\hat{b}(\omega, t)$は時刻$t$における,周波数モード$\omega$の消滅演算子であり,
      \begin{align}
        \qty[\hat{b}(\omega, t), \hat{b}^{\dag}(\omega', t)] = \delta(\omega - \omega')\quad \qty[\hat{b}(\omega, t), \hat{b}(\omega', t)] = 0
      \end{align}
      であるとする.
      また,$\hat{c}$は着目系の演算子のうち,熱浴と相互作用するものである.
      さて,熱浴の演算子$\hat{b}(\omega, t)$や,着目系の任意の物理量$\hat{a}(t)$の時間発展を考えよう.
      Heisenbergの方程式より,
      \begin{align}
        \pdv{\hat{b}}{t}(\omega, t) =& -\frac{\i}{\hbar}\qty[\hat{b}(\omega, t), \hat{H}(t)] \\ 
        =& -\frac{\i}{\hbar}\qty[\hat{b}(\omega, t), \hat{H}_{\r{B}}(t)] - \frac{\i}{\hbar}\qty[\hat{b}(\omega, t), \hat{H}_{\r{int}}(t)] \\ 
        =& -\frac{\i}{\hbar}\int_{-\infty}^{\infty}\dd{\omega'} \hbar\omega'\qty(\hat{b}(\omega, t)\hat{b}^{\dag}(\omega', t)\hat{b}(\omega', t) - \hat{b}^{\dag}(\omega', t)\hat{b}(\omega', t)\hat{b}(\omega, t)) \\ 
        &- \frac{\i}{\hbar}\int_{-\infty}^{\infty}\dd{\omega'} \hbar\i\kappa(\omega')\qty(\hat{b}(\omega, t)\hat{b}^{\dag}(\omega', t)\hat{c}(t) - \hat{b}^{\dag}(\omega', t)\hat{c}(t)\hat{b}(\omega, t)) \\ 
        =& -\i\omega\hat{b}(\omega, t) + \kappa(\omega)\hat{c}(t) \\ 
        \dv{\hat{a}}{t}(t) =& -\frac{\i}{\hbar}\qty[\hat{b}(\omega, t), \hat{H}(t)] \\ 
        =& -\frac{\i}{\hbar}\qty[\hat{a}(t), \hat{H}_{\r{sys}}(t)] - \frac{\i}{\hbar}\qty[\hat{a}(t), \hat{H}_{\r{int}}(t)] \\ 
        =& -\frac{\i}{\hbar}\qty[\hat{a}(t), \hat{H}_{\r{sys}}(t)] + \int_{-\infty}^{\infty}\kappa(\omega)\qty(\hat{b}^{\dag}(\omega, t)\qty[\hat{a}(t), \hat{c}(t)] - \qty[\hat{a}(t), \hat{c}^{\dag}(t)]\hat{b}(\omega, t))
      \end{align}
      である.
      まず,$\hat{b}(\omega, t)$に関する非斉次微分方程式,
      \begin{align}
        \pdv{\hat{b}}{t}(\omega, t) =& -\i\omega\hat{b}(\omega, t) + \kappa(\omega)\hat{c}(t)
      \end{align}
      を未定係数法で解く.
      斉次式は,
      \begin{align}
        \pdv{\hat{b}}{t}(\omega, t) = -\i\omega\hat{b}(\omega, t)
      \end{align}
      であるので,斉次解は,
      \begin{align}
        \hat{b}(\omega, t) = \e^{-\i\omega(t - t_0)}\hat{b}_0(\omega),\quad \hat{b}_0(\omega) \coloneqq \hat{b}(\omega, t_0)
      \end{align}
      である.
      未定係数法では,時間に依存しない$\hat{b}_0(\omega)$を時間依存する項とみなして,元の微分方程式に代入するのであった.
      すると,
      \begin{align}
        \e^{-\i\omega(t - t_0)}\pdv{\hat{b}_0}{t} (\omega, t) - \i\omega(t - t_0)\e^{-\i\omega(t - t_0)}\hat{b}_0(\omega, t) =& -\i\omega\e^{-\i\omega(t - t_0)}\hat{b}_0(\omega) + \kappa(\omega)\hat{c}(t) \\ 
        \iff \pdv{\hat{b}_0}{t} (\omega, t) =& \kappa(\omega)\e^{\i\omega(t - t_0)}\hat{c}(t) \\ 
        \implies \hat{b}_0(\omega, t) =& \kappa(\omega)\int_{t_0}^{t}\dd{t'}\e^{\i\omega(t' - t_0)}\hat{c}(t')
      \end{align}
      と書ける.
      よって,
      \begin{align}
        \hat{b}(\omega, t) =& \e^{-\i\omega(t - t_0)}\hat{b}_0(\omega) + \e^{-\i\omega(t - t_0)}\kappa(\omega)\int_{t_0}^{t}\dd{t'}\e^{\i\omega(t' - t_0)}\hat{c}(t') \\ 
        \iff \hat{b}(\omega, t) =& \e^{-\i\omega(t - t_0)}\hat{b}_0(\omega) + \kappa(\omega)\int_{t_0}^{t}\dd{t'}\e^{-\i\omega(t - t')}\hat{c}(t') \label{PhysRevA.31.3761-b-simple-representation}
      \end{align}
      である.
      $\hat{a}(t)$に関する微分方程式は,\refe{PhysRevA.31.3761-b-simple-representation}を用いて,
      \begin{align}
        \dv{\hat{a}}{t}(t) = -\frac{\i}{\hbar}\qty[\hat{a}(t), \hat{H}_{\r{sys}}(t)] &+ \int_{-\infty}^{\infty}\dd{\omega}\kappa(\omega)\qty{\e^{\i\omega(t - t_0)}\hat{b}^{\dag}_0(\omega)\qty[\hat{a}(t), \hat{c}(t)] - \e^{-\i\omega(t - t_0)}\qty[\hat{a}(t), \hat{c}^{\dag}(t)]\hat{b}_0(\omega)} \\ 
        &+ \int_{-\infty}^{\infty}\dd{\omega}\qty[\kappa(\omega)]^2\int_{t_0}^{t}\dd{t'}\qty(\e^{\i\omega(t - t')}\hat{c}^{\dag}(t')\qty[\hat{a}(t), \hat{c}(t)] - \qty[\hat{a}(t), \hat{c}^{\dag}(t)]\e^{-\i\omega(t - t')}\hat{c}(t')) \label{PhysRevA.31.3761-a-pdv}
      \end{align}
      となる.
      \par
      \refe{PhysRevA.31.3761-a-pdv}に第1 Markov近似を行う.
      すなわち,
      \begin{align}
        \kappa(\omega) = \sqrt{\frac{\gamma}{2\pi}}
      \end{align}
      とする.
      \par
      ところで,$1$をFourier変換すると,
      \begin{align}
        \int_{-\infty}^{\infty}\dd{\omega}\e^{-\i\omega(t - t')} = 2\pi\delta(t - t')
      \end{align}
      となる.
      また,デルタ函数は偶函数であるから,
      \begin{align}
        \int_{t_0}^{t}\dd{t'}c(t')\delta(t - t') =& \int_{0}^{t - t_0}\dd{\tau}c(t - \tau)\delta(\tau) \\ 
        =& \int_{0}^{\infty}\dd{\tau}\delta(t - \tau)\delta(\tau) \\ 
        =& \frac{1}{2}\int_{-\infty}^{\infty}\dd{\tau}\delta(t - \tau)\delta(\tau) \\ 
        =& \frac{1}{2}c(t)
      \end{align}
      となる.
      また,初期状態での熱浴の演算子$\hat{b}_0(\omega) = \hat{b}(\omega, 0)$のFourier変換を,
      \begin{align}
        \hat{b}_{\r{in}}(t) \coloneqq \frac{1}{\sqrt{2\pi}}\int_{-\infty}^{\infty}\dd{\omega}\e^{-\i\omega(t - t_0)}\hat{b}_0
      \end{align}
      と定義する.
      これらの記法を用いると\refe{PhysRevA.31.3761-a-pdv}は,
      \begin{align}
        \dv{\hat{a}}{t}(t) =& -\frac{\i}{\hbar}\qty[\hat{a}(t), \hat{H}_{\r{sys}}(t)] + \int_{-\infty}^{\infty}\dd{\omega}\kappa(\omega)\qty{\e^{\i\omega(t - t_0)}\hat{b}^{\dag}_0(\omega)\qty[\hat{a}(t), \hat{c}(t)] - \e^{-\i\omega(t - t_0)}\qty[\hat{a}(t), \hat{c}^{\dag}(t)]\hat{b}_0(\omega)} \\ 
        &+ \int_{-\infty}^{\infty}\dd{\omega}\qty[\kappa(\omega)]^2\int_{t_0}^{t}\dd{t'}\qty(\e^{\i\omega(t - t')}\hat{c}^{\dag}(t')\qty[\hat{a}(t), \hat{c}(t)] - \qty[\hat{a}(t), \hat{c}^{\dag}(t)]\e^{-\i\omega(t - t')}\hat{c}(t')) \\ 
        =& -\frac{\i}{\hbar}\qty[\hat{a}(t), \hat{H}_{\r{sys}}(t)] + \sqrt{\frac{\gamma}{2\pi}}\qty[\qty{\int_{-\infty}^{\infty}\dd{\omega}\e^{\i\omega(t - t_0)}\hat{b}^{\dag}_0}\qty[\hat{a}(t), \hat{c}(t)] - \qty[\hat{a}(t), \hat{c}^{\dag}(t)]\qty{\int_{-\infty}^{\infty}\dd{\omega}\e^{-\i\omega(t - t_0)}\hat{b}^{\dag}_0}] \\ 
        &+ \frac{\gamma}{2\pi}\qty[\int_{t_0}^{t}\dd{t'}\qty{\int_{-\infty}^{\infty}\dd{\omega}\e^{\i\omega(t - t')}}\hat{c}^{\dag}(t')\qty[\hat{a}(t), \hat{c}(t)] - \int_{t_0}^{t}\dd{t'}\qty{\int_{-\infty}^{\infty}\dd{\omega}\e^{-\i\omega(t - t')}}\qty[\hat{a}(t), \hat{c}(t)]\hat{c}(t')] \\ 
        =& -\frac{\i}{\hbar}\qty[\hat{a}(t), \hat{H}_{\r{sys}}(t)] + \sqrt{\gamma}\qty[\hat{b}_{\r{in}}(t)\qty[\hat{a}(t), \hat{c}(t)] - \qty[\hat{a}(t), \hat{c}^{\dag}(t)]\hat{b}^{\dag}_{\r{in}}(t)] \\ 
        &+ \gamma\qty[\int_{t_0}^{t}\dd{t'}\delta(t - t')\hat{c}^{\dag}(t')\qty[\hat{a}(t), \hat{c}(t)] - \int_{t_0}^{t}\dd{t'}\delta(t - t')\qty[\hat{a}(t), \hat{c}(t)]\hat{c}(t')] \\ 
        =& -\frac{\i}{\hbar}\qty[\hat{a}(t), \hat{H}_{\r{sys}}(t)] + \sqrt{\gamma}\hat{b}_{\r{in}}(t)\qty[\hat{a}(t), \hat{c}(t)] - \sqrt{\gamma}\qty[\hat{a}(t), \hat{c}^{\dag}(t)]\hat{b}^{\dag}_{\r{in}}(t) + \frac{\gamma}{2}\hat{c}^{\dag}(t)\qty[\hat{a}(t), \hat{c}(t)] - \frac{\gamma}{2}\qty[\hat{a}(t), \hat{c}(t)]\hat{c}(t) \\ 
        =& -\frac{\i}{\hbar}\qty[\hat{a}(t), \hat{H}_{\r{sys}}(t)] - \qty[\hat{a}(t), \hat{c}^{\dag}(t)]\qty(\sqrt{\gamma}\hat{b}_{\r{in}} + \frac{\gamma}{2}\hat{c}) + \qty(\sqrt{\gamma}\hat{b}^{\dag}_{\r{in}} + \frac{\gamma}{2}\hat{c}^{\dag})\qty[\hat{a}(t), \hat{c}(t)] 
      \end{align}
      となる.
      よって,$\hat{a}(t)$の時間発展は,
      \begin{align}
        \dv{a}{t} = -\frac{\i}{\hbar}\qty[\hat{a}(t), \hat{H}_{\r{sys}}(t)] - \qty[\hat{a}(t), \hat{c}^{\dag}(t)]\qty(\sqrt{\gamma}\hat{b}_{\r{in}}(t) + \frac{\gamma}{2}\hat{c}(t)) + \qty(\sqrt{\gamma}\hat{b}^{\dag}_{\r{in}}(t) + \frac{\gamma}{2}\hat{c}^{\dag}(t))\qty[\hat{a}(t), \hat{c}(t)] \label{PhysRevA.31.3761-a-markov-approximated}
      \end{align}
      と書ける.
      \par
      第1 Markov近似を行った\refe{PhysRevA.31.3761-a-pdv}は次の4つの特徴を持つ.
      \begin{enumerate}
        \item 着目系に関して減衰項が存在する.
          着目系が調和振動子モデルのハミルトニアンを持ち,その周波数は$\omega_0$の整数倍であるとする.
          熱浴と相互作用する着目系の演算子は$\hat{a}$であり,またそれだけのとき$\hat{c}(t) = \hat{a}(t)$として\refe{PhysRevA.31.3761-a-markov-approximated}は,
          \begin{align}
            \dv{a}{t} =& -\frac{\i}{\hbar}\qty[\hat{a}(t), \hbar\omega_0\hat{a}^{\dag}(t)\hat{a}(t)] - \qty[\hat{a}(t), \hat{a}^{\dag}(t)]\qty(\sqrt{\gamma}\hat{b}_{\r{in}}(t) + \frac{\gamma}{2}\hat{a}(t)) + \qty(\sqrt{\gamma}\hat{b}^{\dag}_{\r{in}}(t) + \frac{\gamma}{2}\hat{a}^{\dag}(t))\qty[\hat{a}(t), \hat{a}(t)] \\ 
            \iff \dv{a}{t} =& -\frac{\i}{\hbar}\hat{a}(t) - \sqrt{\gamma}\hat{b}_{\r{in}}(t) - \frac{\gamma}{2}\hat{a}(t)
          \end{align}
          となり,第1項は周波数$\hat{\omega}_0$で振動する項,第2項は減衰する項である.
        \item $\hat{b}_\r{in}$はノイズ項である.
          初期状態$\hat{b}_{\r{in}}$に依存せず,\refe{PhysRevA.31.3761-a-pdv}は成立する.
      \end{enumerate}
      伊藤積分以降,わからない.
      数理物理の分野になるから深追いしないことにした.